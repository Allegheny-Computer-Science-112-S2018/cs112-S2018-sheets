\documentclass[11pt]{article}

% NOTE: The "Edit" sections are changed for each assignment

% Edit these commands for each assignment

\newcommand{\assignmentduedate}{March 13}
\newcommand{\assignmentassignedate}{March 6}
\newcommand{\assignmentnumber}{Seven}

\newcommand{\labyear}{2018}
\newcommand{\labday}{Tuesday}
\newcommand{\labtime}{2:30 pm}

\newcommand{\assigneddate}{Assigned: \labday, \assignmentassignedate, \labyear{} at \labtime{}}
\newcommand{\duedate}{Due: \labday, \assignmentduedate, \labyear{} at \labtime{}}

% Edit these commands to give the name to the main program

\newcommand{\mainprogram}{\lstinline{Experiment}}
\newcommand{\mainprogramsource}{\lstinline{src/main/java/labseven/experiment/Experiment.java}}

% Edit these commands to give the main program's output details

\newcommand{\mainprogramoutput}{four}

% Edit these commands to give the name to the test suite

\newcommand{\testprogram}{\lstinline{TestExperiment}}
\newcommand{\testprogramsource}{\lstinline{src/test/java/labseven/TestExperiment.java}}

% Edit this commands to describe key deliverables

\newcommand{\reflection}{\lstinline{writing/reflection.md}}

% Commands to describe key development tasks

% --> Running gatorgrader.sh
\newcommand{\gatorgraderstart}{\command{./gatorgrader.sh --start}}
\newcommand{\gatorgradercheck}{\command{./gatorgrader.sh --check}}

% --> Compiling and running and testing program with gradle
\newcommand{\gradlebuild}{\command{gradle build}}
\newcommand{\gradletest}{\command{gradle test}}
\newcommand{\gradlerun}{\command{gradle run}}

% Commands to describe key git tasks

% NOTE: Could be improved, problems due to nesting

\newcommand{\gitcommitfile}[1]{\command{git commit #1}}
\newcommand{\gitaddfile}[1]{\command{git add #1}}

\newcommand{\gitadd}{\command{git add}}
\newcommand{\gitcommit}{\command{git commit}}
\newcommand{\gitpush}{\command{git push}}
\newcommand{\gitpull}{\command{git pull}}

\newcommand{\gitcommitmainprogram}{\command{git commit src/main/java/labseven/experiment/Experiment.java -m "Your
descriptive commit message"}}

% Use this when displaying a new command

\newcommand{\command}[1]{``\lstinline{#1}''}
\newcommand{\program}[1]{\lstinline{#1}}
\newcommand{\url}[1]{\lstinline{#1}}
\newcommand{\channel}[1]{\lstinline{#1}}
\newcommand{\option}[1]{``{#1}''}
\newcommand{\step}[1]{``{#1}''}

\usepackage{pifont}
\newcommand{\checkmark}{\ding{51}}
\newcommand{\naughtmark}{\ding{55}}

\usepackage{listings}
\lstset{
  basicstyle=\small\ttfamily,
  columns=flexible,
  breaklines=true
}

\usepackage{fancyhdr}

\usepackage[margin=1in]{geometry}
\usepackage{fancyhdr}

\pagestyle{fancy}

\fancyhf{}
\rhead{Computer Science 112}
\lhead{Laboratory Assignment \assignmentnumber{}}
\rfoot{Page \thepage}
\lfoot{\duedate}

\usepackage{titlesec}
\titlespacing\section{0pt}{6pt plus 4pt minus 2pt}{4pt plus 2pt minus 2pt}

\newcommand{\labtitle}[1]
{
  \begin{center}
    \begin{center}
      \bf
      CMPSC 112\\Introduction to Computer Science II\\
      Spring 2018\\
      \medskip
    \end{center}
    \bf
    #1
  \end{center}
}

\begin{document}

\thispagestyle{empty}

\labtitle{Laboratory \assignmentnumber{} \\ \assigneddate{} \\ \duedate{}}

\section*{Objectives}

To continue to practice using GitHub to access the files for an assignment. You
will complete a programming project using source code provided in the textbook
and by the course instructor ultimately implementing a framework for
experimentally evaluating two algorithms for arithmetic computation.
Specifically, you will study both an iterative and a recursive implementation of
Java programs that calculate numbers in the Fibonacci sequence. You will also
practice creating a data table and calculating an order of growth ratio, honing
your ability to analytically and empirically evaluate the worst-case time
complexity of an algorithm. Finally, you will continue to learn how to implement
and test a Java program and to write a Markdown file, practicing how to use an
automated grading tool to assess your progress towards correctly completing the
project.

\section*{Suggestions for Success}

\begin{itemize}
  \setlength{\itemsep}{0pt}

\item {\bf Use the laboratory computers}. The computers in this laboratory feature specialized software for completing
  this course's laboratory and practical assignments. If it is necessary for you to work on a different machine, be sure
  to regularly transfer your work to a laboratory machine so that you can check its correctness. If you cannot use a
  laboratory computer and you need help with the configuration of your own laptop, then please carefully explain its
  setup to a teaching assistant or the course instructor when you are asking questions.

\item {\bf Follow each step carefully}. Slowly read each sentence in the assignment sheet, making sure that you
  precisely follow each instruction. Take notes about each step that you attempt, recording your questions and ideas and
  the challenges that you faced. If you are stuck, then please tell a teaching assistant or instructor what assignment
  step you recently completed.

\item {\bf Regularly ask and answer questions}. Please log into Slack at the start of a laboratory or practical session
  and then join the appropriate channel. If you have a question about one of the steps in an assignment, then you can
  post it to the designated channel. Or, you can ask a student sitting next to you or talk with a teaching assistant or
  the course instructor.

\item {\bf Store your files in GitHub}. As in previous laboratory assignments, you will be responsible for storing
  all of your files (e.g., Java source code and Markdown-based writing) in a Git repository using GitHub Classroom.
  Please verify that you have saved your source code in your Git repository by using \command{git status} to ensure that
  everything is updated. You can see if your assignment submission meets the established correctness requirements by
  using the provided checking tools on your local computer and in checking the commits in GitHub.

\item {\bf Keep all of your files}. Don't delete your programs, output files, and written reports after you submit them
  through GitHub; you will need them again when you study for the quizzes and examinations and work on the other
  laboratory, practical, and final project assignments.

\item {\bf Back up your files regularly}. All of your files are regularly backed-up to the servers in the Department of
  Computer Science and, if you commit your files regularly, stored on GitHub. However, you may want to use a flash
  drive, Google Drive, or your favorite backup method to keep an extra copy of your files on reserve. In the event of
  any type of system failure, you are responsible for ensuring that you have access to a recent backup copy of all your
  files.

\item {\bf Explore teamwork and technologies}. While certain aspects of the laboratory assignments will be challenging
  for you, each part is designed to give you the opportunity to learn both fundamental concepts in the field of computer
  science and explore advanced technologies that are commonly employed at a wide variety of companies. To explore and
  develop new ideas, you should regularly communicate with your team and/or the teaching assistants and tutors.

\item {\bf Hone your technical writing skills}. Computer science assignments require to you write technical
  documentation and descriptions of your experiences when completing each task. Take extra care to ensure that your
  writing is interesting and both grammatically and technically correct, remembering that computer scientists must
  effectively communicate and collaborate with their team members and the tutors, teaching assistants, and course
  instructor.

\item {\bf Review the Honor Code on the syllabus}. While you may discuss your assignments with others, copying source
  code or writing is a violation of Allegheny College's Honor Code.

\end{itemize}

\section*{Reading Assignment}

If you have not done so already, please read all of the relevant ``GitHub
Guides'', available at \url{https://guides.github.com/}, that explain how to use
many of the features that GitHub provides. In particular, please make sure that
you have read guides such as ``Mastering Markdown'' and ``Documenting Your
Projects on GitHub''; each of them will help you to understand how to use both
GitHub and GitHub Classroom. To do well on this laboratory assignment, you
should also read Section 1.5.2 in the textbook, paying particularly close
attention to the material about iteration constructs. You should also review
Sections 4.1 through 4.3 and read Sections 5.1 through 5.5, focusing on Code
Fragment 5.13 and the text that describes and analyzes it. Please see the course
instructor or one of the teaching assistants if you have questions about these
reading assignments.

\section*{Accessing the Laboratory Assignment on GitHub}

To access the laboratory assignment, you should go into the
\channel{\#announcements} channel in our Slack team and find the announcement
that provides a link for it. Copy this link and paste it into a web browser.
Now, you should accept the assignment and see that GitHub Classroom created a
new GitHub repository for you to access the assignment's starting materials and
to store the completed version of your assignment. Specifically, to access your
new GitHub repository for this assignment, please click the green ``Accept''
button and then click the link that is prefaced with the label ``Your assignment
has been created here''. If you accepted the assignment and correctly followed
these steps, you should have created a GitHub repository with a name like
``Allegheny-Computer-Science-112-Spring-2018/computer-science-112-spring-2018-lab-7-gkapfham''.
Unless you provide the instructor with documentation of the extenuating
circumstances that you are facing, not accepting the assignment means that you
automatically receive a failing grade for it.

Before you move to the next step of this assignment, please make sure that you
read all of the content on the web site for your new GitHub repository, paying
close attention to the technical details about the commands that you will type
and the output that your program must produce. Now you are ready to download the
starting materials to your laboratory computer. Click the ``Clone or download''
button and, after ensuring that you have selected ``Clone with SSH'', please
copy this command to your clipboard. To enter into your course directory you
should now type \command{cd cs112S2018}. By typing \command{git clone} in your
terminal and then pasting in the string that you copied from the GitHub site you
will download all of the code for this assignment. For instance, if the course
instructor ran the \command{git clone} command in the terminal, it would look
like:

\begin{lstlisting}
  git clone git@github.com:Allegheny-Computer-Science-112-S2018/computer-science-112-spring-2018-lab-7-gkapfham.git
\end{lstlisting}

After this command finishes, you can use \command{cd} to change into the new
directory. If you want to \step{go back} one directory from your current
location, then you can type the command \command{cd ..}. Please continue to use
the \command{cd} and \command{ls} commands to explore the files that you
automatically downloaded from GitHub. What files and directories do you see?
What do you think is their purpose? Spend some time exploring, sharing your
discoveries with a neighbor and a \mbox{teaching assistant}. Specifically, each
student should draw a diagram to show the relationship between the Java classes
provided for this laboratory assignment. Make sure that you have the instructor
check your project diagram.

\section*{Evaluating the Performance of Recursion and Iteration}

There are many \command{TODO} markers inside of the provided source code.
Ultimately, it is your responsibility to read each of these and provide the
required source code. With that said, the remainder of this section will provide
some pointers to help you to implement the needed code. Specifically, this
assignment asks you to run experiments that produce output like that which
Figure~\ref{fig:output} provides. In particular, you will implement, test, and
experimentally study a recursive and an iterative implementation of a method
that can calculate specific values from the Fibonacci sequence. This well-known
sequence includes the numbers in the integer sequence that develops in the
following fashion: $0, 1, 1, 2, 3, 5, 8, 13, 21, 34, 55, 89, \ldots$. More
formally, we can define the {\em n}-th Fibonacci number, denoted $F_n$, by the
equation $F_n = F_{n-1} + F_{n-2}$. Notably, this equation is recursive in
nature because the {\em n}-th value is defined in terms of the two previous
values.

% There are several \command{TODO} markers inside of the \program{RunCampaign}
% class. First, you need to implement a method that can randomly generate an
% \program{int[]} array. You can refer to the test suite for this laboratory
% assignment and from previous assignments to learn how to write this method. The
% next step is one of the most challenging for this assignment! Please review all
% of the \command{TODO} markers in the \program{run} method of the
% \program{RunCampaign} class and notice that you need to first call the method
% that generates the input array at the specified size. Next, you need to sort
% this array using the \program{sort} method provided by the \program{Sorter} that
% is a parameter to this method. Please make sure that you adopt the strategy in
% Code Fragment 4.1 to time the sorting algorithm. Finally, you need to record the
% timing results in the data table, ensure that the experiment campaign will
% continue with the next doubled size for the input array, and then go to the next
% round of the experiment campaign.

% You should also note that this assignment requires you to complete the
% implementation of a \program{ResultsTable} that can store the results from each
% round of your experiment campaign. To start your implementation of this class,
% please finish its constructor by creating the two-dimensional array and
% indicating that the first row of the table will start at zero. Finally, you must
% provide an implementation of the \program{public void addResult(long size, long
% timing)} method that will record a new row of data in the \program{ResultsTable}
% while also calculating the ``order of growth ratio'' between the current data
% value and the previous value. This step is asking you to write source code that
% can calculate the third row of the table evident in Figure~\ref{fig:output}.
% This ratio will help you to understand the algorithm's efficiency. As you are
% implementing this method, make sure that you take care to set the ratio for the
% first row to zero and avoid dividing by zero when computing this value.

% It is worth pointing out that your textbook contains several useful insights
% into the pattern that you should follow when thinking about the running time of
% the two provided sorting algorithms. For instance, when describing the results
% from running an experiment, page 153 notes that ``[A]s the value of $n$ is
% doubled, the running time of {\tt repeat1} typically increases more than
% fourfold.'' What does this suggest about the likely ``worst-case time
% complexity'' of the {\tt repeat1} method? How can you apply this intuition to
% analyze the results that you collect when running an experiment?

% Additionally, page 172 includes the following statement when describing the
% performance of {\tt repeat2}: ``the running times in that table $\ldots$
% demonstrate a trend of approximately doubling each time the problem size
% doubles.'' Again, what would this observation suggest about the likely
% worst-case time complexity of {\tt repeat2}?

Remember, if you want to \step{build} your program you can type the command
\gradlebuild{} in your terminal, thereby causing the Java compiler to check your
program for errors and get it ready to run. If you notice that some of the test
cases do not pass, please improve your implementation until all of the tests
pass and your program's output looks similar to that which is provided in
Figure~\ref{fig:output}. Once the program runs and the tests pass, please
reflect on this process. What step did you find to be the most challenging? Why?
You should write your reflections in a file, called \reflection{}, that uses the
Markdown writing language. You can learn more about Markdown by viewing the
aforementioned GitHub guide. To complete this part of the assignment, you should
write one paragraph that reports on your experiences. Your \reflection{} file
should also have answers to the questions about \mainprogram{}'s implementation,
testing, and use.

\begin{figure}[t]
  \centering
  \begin{verbatim}
  Results of an experiment campaign with IterativeFibonacciComputation:

  Size (#)        Timing (ms)     Ratio (#)
  1               0               0
  2               0               0
  4               0               0
  8               0               0
  16              0               0
  32              0               0
  64              0               0
  128             0               0
  256             0               0
  512             0               0
  1024            0               0
  2048            0               0
  4096            0               0
  8192            1               0
  16384           0               0
  32768           1               0
  65536           2               2
  131072          1               1
  262144          1               1
  524288          1               1
  1048576         1               1
  2097152         1               1
  4194304         2               2
  \end{verbatim}
  \vspace*{-.35in}
  \caption{A Portion of the Expected Output of the \mainprogram{} Program.}~\label{fig:output}
  \vspace*{-.25in}
\end{figure}

\section*{Checking the Correctness of Your Program and Writing}

As verified by the Checkstyle tool, the source code for the \mainprogram{} and
the other Java files must adhere to all of the requirements in the Google Java
Style Guide available at
\url{https://google.github.io/styleguide/javaguide.html}. The Markdown file that
contains your reflection must adhere to the standards described in the Markdown
Syntax Guide \url{https://guides.github.com/features/mastering-markdown/}.
Finally, your \reflection{} file should adhere to the Markdown standards
established by the \step{Markdown linting} tool available at
\url{https://github.com/markdownlint/markdownlint/}. Instead of requiring you to
manually check that your deliverables adhere to these industry-accepted
standards, GatorGrader makes it easy for you to automatically check if your
submission meets the standards for correctness. For instance, GatorGrader will
run the provided JUnit tests, check to ensure that two source code files contain
the right number of \program{println} statements, and see that you commit to
GitHub a sufficient number of times when completing this assignment. The focus
for this assignment is checking that all of the provided tests correctly pass.

To get started with the use of GatorGrader, type the command \gatorgraderstart{}
in your terminal window. Once this step completes you can type
\gatorgradercheck{}. If your work does not meet all of the assignment's
requirements, then you will see the following output in your terminal:
\command{Overall, are there any mistakes in the assignment? Yes}. If you do have
mistakes in your assignment, then you will need to review GatorGrader's output,
find the mistake, and try to fix it. Once your program is building correctly,
fulfilling at least some of the assignment's requirements, you should transfer
your files to GitHub using the \gitcommit{} and \gitpush{} commands. For
example, if you want to signal that the \mainprogramsource{} file has been
changed and is ready for transfer to GitHub you would first type
\gitcommitmainprogram{} in your terminal, followed by typing \gitpush{} and
checking to see that the transfer to GitHub is successful. If you notice that
transferring your code or writing to GitHub did not work, then please try to
determine why, asking a teaching assistant or the course instructor for
assistance, if necessary.

After the course instructor enables \step{continuous integration} with a system called Travis CI, when you use the
\gitpush{} command to transfer your source code to your GitHub repository, Travis CI will initialize a \step{build} of
your assignment, checking to see if it meets all of the requirements. If both your source code and writing meet all of
the established requirements, then you will see a green \checkmark{} in the listing of commits in GitHub after awhile.
If your submission does not meet the requirements, a red \naughtmark{} will appear instead. The instructor will reduce a
student's grade for this assignment if the red \naughtmark{} appears on the last commit in GitHub immediately before the
assignment's due date. Yet, if the green \checkmark{} appears on the last commit in your GitHub repository, then you
satisfied all of the main checks, thereby allowing the course instructor to evaluate other aspects of your source code
and writing, as further described in the \step{Evaluation} section of this assignment sheet. Unless you provide the
instructor with documentation of the extenuating circumstances that you are facing, no late work will be considered
towards your grade for this laboratory assignment.

\section*{Summary of the Required Deliverables}

\noindent Students do not need to submit printed source code or technical
writing for any assignment in this course. Instead, this assignment invites you
to submit, using GitHub, the following deliverables.

\begin{enumerate}

  \setlength{\itemsep}{0in}

\item Stored in a Markdown file called \reflection{}, a one-paragraph response,
  for each of the stated questions. Make sure to include the requested results
  in a fenced code block.

\item A properly documented and correct version of all the Java source files
  (i.e., Fibonacci methods, experiments, and tests) that meets the set
  requirements and produces the desired output.

\end{enumerate}

\section*{Evaluation of Your Laboratory Assignment}

Using a report that the instructor shares with you through the commit log in GitHub, you will privately received a grade
on this assignment and feedback on your submitted deliverables. Your grade for the assignment will be a function of the
whether or not it was submitted in a timely fashion and if your program received a green \checkmark{} indicating that it
met all of the requirements. Other factors will also influence your final grade on the assignment. In addition to
studying the efficiency and effectiveness of your Java source code, the instructor will also evaluate the accuracy of
both your technical writing and the comments in your source code. If your submission receives a red \naughtmark{}, the
instructor will reduce your grade for the assignment while still considering the regularity with which you committed to
your GitHub repository and the overall quality of your partially completed work. Please see the instructor if you have
questions about the evaluation of this laboratory assignment.

% \section*{Adhering to the Honor Code}

% In adherence to the Honor Code, students should complete this assignment on an individual basis. While it is appropriate
% for students in this class to have high-level conversations about the assignment, it is necessary to distinguish
% carefully between the student who discusses the principles underlying a problem with others and the student who produces
% assignments that are identical to, or merely variations on, someone else's work. Deliverables (e.g., Java source code or
% Markdown-based technical writing) that are nearly identical to the work of others will be taken as evidence of violating
% the \mbox{Honor Code}. Please see the course instructor if you have questions about this policy.

\end{document}
