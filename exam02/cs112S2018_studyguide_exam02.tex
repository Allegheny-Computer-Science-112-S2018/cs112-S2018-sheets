\documentclass[11pt]{article}

% NOTE: The "Edit" sections are changed for each assignment

% Edit these commands for each assignment

\newcommand{\assignmentduedate}{April 6}
\newcommand{\assignmentassignedate}{April 2}
\newcommand{\assignmentnumber}{One}

\newcommand{\labyear}{2018}
\newcommand{\assignedday}{Monday}
\newcommand{\dueday}{Friday}
\newcommand{\labtime}{1:30 pm}

\newcommand{\assigneddate}{Announced: \assignedday, \assignmentassignedate, \labyear{} at \labtime{}}
\newcommand{\duedate}{Exam: \dueday, \assignmentduedate, \labyear{} at \labtime{}}

% Edit these commands to give the name to the main program

\newcommand{\mainprogram}{\lstinline{DisplayOutput}}
\newcommand{\mainprogramsource}{\lstinline{src/main/java/labone/DisplayOutput.java}}

% Edit this commands to describe key deliverables

\newcommand{\reflection}{\lstinline{writing/reflection.md}}

% Commands to describe key development tasks

% --> Running gatorgrader.sh
\newcommand{\gatorgraderstart}{\command{./gatorgrader.sh --start}}
\newcommand{\gatorgradercheck}{\command{./gatorgrader.sh --check}}

% --> Compiling and running program with gradle
\newcommand{\gradlebuild}{\command{gradle build}}
\newcommand{\gradlerun}{\command{gradle run}}

% Commands to describe key git tasks

% NOTE: Could be improved, problems due to nesting

\newcommand{\gitcommitfile}[1]{\command{git commit #1}}
\newcommand{\gitaddfile}[1]{\command{git add #1}}

\newcommand{\gitadd}{\command{git add}}
\newcommand{\gitcommit}{\command{git commit}}
\newcommand{\gitpush}{\command{git push}}
\newcommand{\gitpull}{\command{git pull}}

\newcommand{\gitcommitmainprogram}{\command{git commit src/main/java/labone/DisplayOutput.java -m "Your
descriptive commit message"}}

% Use this when displaying a new command

\newcommand{\command}[1]{``\lstinline{#1}''}
\newcommand{\program}[1]{\lstinline{#1}}
\newcommand{\url}[1]{\lstinline{#1}}
\newcommand{\channel}[1]{\lstinline{#1}}
\newcommand{\option}[1]{``{#1}''}
\newcommand{\step}[1]{``{#1}''}

\usepackage{pifont}
\newcommand{\checkmark}{\ding{51}}
\newcommand{\naughtmark}{\ding{55}}

\usepackage{listings}
\lstset{
  basicstyle=\small\ttfamily,
  columns=flexible,
  breaklines=true
}

\usepackage{fancyhdr}

\usepackage[margin=1in]{geometry}
\usepackage{fancyhdr}

\pagestyle{fancy}

\fancyhf{}
\rhead{Computer Science 112}
\lhead{Exam \assignmentnumber{}}
\rfoot{Page \thepage}
\lfoot{\duedate}

\usepackage{titlesec}
\titlespacing\section{0pt}{6pt plus 4pt minus 2pt}{4pt plus 2pt minus 2pt}

\newcommand{\guidetitle}[1]
{
  \begin{center}
    \begin{center}
      \bf
      CMPSC 112\\Introduction to Computer Science II\\
      Spring 2018\\
      \medskip
    \end{center}
    \bf
    #1
  \end{center}
}

\begin{document}

\thispagestyle{empty}

\guidetitle{Exam \assignmentnumber{} Study Guide \\ \assigneddate{} \\ \duedate{}}

\section*{Introduction}

\noindent
The exam will be ``closed notes'' and ``closed book'' and it will cover the
following materials. Please review the ``Course Schedule'' on the Web site for
the course to see the content and slides that we have covered to this date.
Students may post questions about this material to our Slack team.

\begin{itemize}

  \itemsep 0in

  \item Chapter One in DSAAJ, all sections (i.e., ``Java Primer'')

  \item Chapter Two in DSAAJ, all sections (i.e., ``Object-Oriented Design'')

  \item Chapter Three in DSAAJ, all sections (i.e., ``Fundamental Data Structures'')

  \item Chapter Four in DSAAJ, skipping Section 4.4 (i.e., ``Algorithm Analysis'')

  \item Chapter Five in DSAAJ, skipping Section 5.6 (i.e., ``Recursion'')

  \item Chapter Seven in DSAAJ, skipping Sections 7.3, 7.5, and 7.6 (i.e., ``Lists and Iterators'')

  \item Using the commands in the terminal window (e.g., \program{cd} and
    \program{ls}); building and running Java programs with Gradle; knowledge of
    the basic commands for using \program{git} and GitHub

  \item Your class notes and the discussion slides available from the course web
    site

  \item Source code and writing for laboratory assignments 1--8 and practical
    assignments 1--6

\end{itemize}

\noindent The exam will be a mix of questions that have a form such as fill in
the blank, short answer, true/false, and completion. The emphasis will be on the
following topics:

\vspace*{-.05in}
\begin{itemize}

  \itemsep 0in

  \item Fundamental concepts in computing and the Java language (e.g.,
    definitions and background)

  \item Practical laboratory techniques (e.g., editing, building, and running
    programs; effectively using files and directories; correctly using GitHub
    through the command-line {\tt git} program)

  \item Understanding Java programs (e.g., given a short, perhaps even one line,
    source code segment written in Java, understand what it does and be able to
    precisely describe its output).

  \item Knowledge of worst-case time complexities, expressed in the ``big-Oh''
    notation, for all of the algorithms provided by the studied data structures
    (e.g., \program{removeLast} for a node-based list).

  \item Composing Java statements and programs, given a description of what
    should be done. Students should be completely comfortable writing short
    source code statements that are in nearly-correct form as Java code. While
    your program may contain small syntactic errors, it is not acceptable to
    ``make up'' features of the Java programming language that do not exist in
    the language itself---so, please do not call a ``{\tt
    solveQuestionThree()}'' method!

\end{itemize}

\noindent No partial credit will be given for questions that are true/false,
completion, or fill in the blank. Minimal partial credit may be awarded for the
questions that require a student to write a short answer. You are strongly
encouraged to write short, precise, and correct responses to all of the
questions. When you are taking the exam, you should do so as a ``point
maximizer'' who first responds to the questions that you are most likely to
answer correctly for full points. Please keep the time limitation in mind as you
are absolutely required to submit the examination at the end of the class period
unless you have written permission for extra time from a member of the Learning
Commons. Students who do not submit their exam on time will have their overall
point total reduced. Finally, students may only reschedule the exam for a
different date or time if they are facing documented extenuating circumstances
that prevent them from attending the scheduled time slot. Please see the course
instructor if you have questions about any of these policies.

\section*{Reminder Concerning the Honor Code}

\noindent Students are required to fully adhere to the Honor Code during the
completion of this exam. More details about the Allegheny College Honor Code are
provided on the syllabus. Students are strongly encouraged to carefully review
the full statement of the Honor Code before taking this exam. If you do not
understand Allegheny College's Honor Code, please schedule a meeting with the
course instructor. The following is a review of Honor Code statement from the
course syllabus:

\vspace*{-.05in}

\begin{quote}
The Academic Honor Program that governs the entire academic program at
Allegheny College is described in the Allegheny Academic Bulletin. The Honor
Program applies to all work that is submitted for academic credit or to meet
non-credit requirements for graduation at Allegheny College. This includes all
work assigned for this class (e.g., examinations, laboratory assignments, and
the final project). All students who have enrolled in the College will work
under the Honor Program.  Each student who has matriculated at the College has
acknowledged the following pledge:
\end{quote}

\vspace*{-.1in}

\begin{quote}
  I hereby recognize and pledge to fulfill my responsibilities, as defined in the Honor Code, and to maintain the
  integrity of both myself and the College community as a whole.
\end{quote}

\section*{Detailed Review of Content}

The listing of topics in the following subsections is not exhaustive; rather, it
serves to illustrate the types of concepts that students should study as they
prepare for the exam. Please see the instructor during office hours if you have
questions about any of the content listed in this section.

\vspace*{-.1in}

\subsection*{Chapter One}

\begin{itemize}

  \item Basic syntax and semantics of the Java programming language
  \item Input(s) and output(s) of the Java compiler and virtual machine
  \item How to use \command{gradle} to build, run, and test a Java program
  \item The base types available for primitive variables in the Java language
  \item How to create and use classes and objects in the Java language
  \item How to define and call methods and constructors of a Java class
  \item The declaration and use of \program{String}, \program{StringBuffer}, and arrays
  \item Type conversion operators and control flow constructs in Java
  \item Software engineering principles as expressed in Java (e.g., packages and
    JUnit tests)
  \item How to write effective test cases in JUnit for various data structures
    and algorithms
  \item How to use the JavaDoc standard to write informative
    comments in Java programs
  \item An understanding of the Java source code supporting a program that
    performs input and output with the console (e.g.,
    \program{System.out.println}) and files (e.g., \program{java.util.Scanner})

\end{itemize}

\vspace*{-.2in}
\subsection*{Chapter Two}

\begin{itemize}

  \setlength{\itemsep}{0.05in}

  \item The goals, principles, and patterns of object-oriented design in the
    Java language
  \item An understanding of the principles known as abstraction, encapsulation, and modularity
  \item How to use inheritance hierarchies to create an ``is a'' relationship
    between Java classes
  \item The meaning and purpose of abstract classes and interfaces in the Java
    programming language
  \item How to create, catch, and handle exceptions thrown in a Java program
  \item How to performing casting to convert a variable from one data type to another
  \item The ways in which generics promote the implementation of reusable Java
    programs

\end{itemize}

\vspace*{-.2in}
\subsection*{Chapter Three}

\begin{itemize}

  \setlength{\itemsep}{0.05in}

  \item How to use arrays to store primitive and reference variables
  \item The algorithms for sorting arrays into ascending and descending order
  \item How to use psuedo random number generators in Java programs
  \item The similarities and differences between one- and two-dimensional arrays
  \item The meaning and purpose of techniques for cloning data structures
  \item The similarities and differences between ``deep'' and ``shallow'' copies
    of arrays
  \item How to use the Java methods to construct ``deep'' and ``shallow'' copies
    of data structures
  \item An understanding of the types of nodes in a \program{SinglyLinkedList} and \program{DoublyLinkedList}
  \item The benefits and trade-offs associated with the \program{SinglyLinkedList} and \program{DoublyLinkedList}
  \item Knowledge of the worst-case time complexity for all methods of node-based structures
  \item Why the \program{removeLast} implementation is inefficient when
    implemented in a \program{SinglyLinkedList}
  \item The meaning of the term ``equivalence testing'' and how it connects to data structures
  \item The ``equivalence relationship'' that must be upheld by, for instance, \program{DoublyLinkedList}
  \item The trade-offs associated with using either arrays or linked lists to implement data structures
  \item The benefits that come from using Java's generics to implement node-based list
    structures

\end{itemize}

\vspace*{-.2in}
\subsection*{Chapter Four}

\begin{itemize}

  \setlength{\itemsep}{0.05in}

  \item A strategy for timing the implementation of an algorithm in the Java
    programming language
  \item The challenges associated with experimentally studying an algorithm's
    performance
  \item A comprehensive understanding of why to use a doubling experiment to
    study efficiency
  \item An understanding of all of the steps and source code needed to conduct a
    doubling experiment
  \item The challenges associated with conduct an experimental evaluation of an
    algorithm
  \item The meaning and purpose of the terms ``basic operation'' and ``psuedo code''
  \item An intuitive understanding of best-, worst-, and average-case analytical
    evaluations
  \item Knowledge of the seven functions used to characterize an algorithm's
    complexity class
  \item The relationship between an order-of-growth ratio and the worst-case
    time complexity
  \item The meaning of the ``Big-Oh'' notation used during the
    analytical evaluation of algorithms
  \item Knowledge of the patterns in psuedo code that suggest certain
    worst-case time complexities
  \item How to intuitively prove the worst-case time complexity of an algorithm
    using psuedo code

\end{itemize}

\vspace*{-.2in}
\subsection*{Chapter Five}

\begin{itemize}

  \setlength{\itemsep}{0.05in}

  \item How to use a recursive approach to solving a problem (e.g., the base
    and recursive cases)
  \item Knowledge of recursive algorithms for arithmetic computation (e.g., factorial
    and Fibonacci)
  \item How to determine and justify the worst-case time complexity for
    the recursive factorial method
  \item The similarities and differences of algorithms that have linear, binary, and multiple
    recursion
  \item An understanding of how different recursive strategies influence the
    efficiency of an algorithm
  \item A precise understanding of what it means for a recursive algorithm to
    ``run amok''
  \item The ability to explain why the ``traditional'' recursive Fibonacci
    method is inefficient
  \item The way in which the Java programming language's virtual machine runs recursive methods
  \item The ability to draw a ``call tree'' diagram to explain the
    execution of a recursive method
  \item An understanding of why a recursive method can cause a ``stack
    overflow'' error
  \item The meaning and behavior of a Java method that has an infinite recursion

\end{itemize}

\vspace*{-.2in}
\subsection*{Chapter Seven}

\begin{itemize}

  \setlength{\itemsep}{0.05in}

  \item A fundamental understanding of the meaning and purposes of a list
    abstract data type

  \item An approach to implementing and testing an array-based version of the
    list data structure

  \item The benefits that come from using Java's generics to implement
    array-based list structures

  \item The four steps that a dynamic array must perform when the internal
    fixed-sized array is full

  \item How the \program{StringBuilder} uses a dynamic array to implement
    a mutable string that is efficient

  \item The impact that the use of an \program{ArrayList} has on the memory
    usage of a Java program

  \item The trade-offs associated with using either arrays or linked lists to implement data structures

  \item Knowledge of the worst-case time complexities for all of the methods
    of the dynamic array

  \item Knowledge of how to use an \program{Iterator} in a Java program that
    creates and manipulates lists

  \item A basic understanding of the data structures that the Java programming
    language provides in the ``Collections'' framework (e.g.,
    \program{java.util.ArrayList} and \program{java.util.Vector})

\end{itemize}

\end{document}
